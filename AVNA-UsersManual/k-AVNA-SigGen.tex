\section{Audio Signal Generator(s)}
xxx
\subsection{Description}
xxx      


%MOVE >>>>>>>>>>>>>>>>>>>>>>>>>>
%A couple of minor corrections.  In the example, the comments on the SigGen settings have the wrong amplitude levels.   The commands are OK.  The correct comments are \\ \\
 %\texttt{SIGGEN 1 1 1234.56 0.1 3}  // SG \#1 as a triangle wave 0.1 v p-p \\
 %\texttt{SIGGEN 4 1 0.0 0.003 0}  // SG \#4 Gaussian White Noise, $\sigma = 0.003$\\

\subsection{Instructions}
\subsubsection{Waveforms}
There are three signal generators and a fourth Gaussian White Noise Generator. The sum of these waveforms appears at the left or "impedance" terminals.  These are not available when the AVNA instrument is operating, that is, when impedance or transmission measurements are being made.

The signal generators can produce either sine or square waves when under direct screen control.  The frequency can be specified from 10 to 40000 Hz in 1 Hz steps on-screen.  The amplitude is specified by the peak-to-peak value in volts.  The output impedance is 50-Ohms and the voltage displayed is that delivered to a 50-Ohm load. When open-circuited, the voltage will be double the displayed value.  For convenience, there is provision to turn each generator on and off while leaving the settings alone.

For sine waves, the waveform is excellent up to about 5/12th of the sample frequency, above which the amplitude drops off.   That corresponds to 40 kHz for the highest 96 kHz sample frequency.  For non-sine waves harmonics of the fundamental frequency are needed to build the waveform.  The missing pieces, above 5/12 of the sample frequency, cause the waveform becomes distorted.  For waveforms like the square wave, this can be an important limitation and suggests keeping the frequency well below half the sample frequency.  In general, it pays to look at the waveform on an oscilloscope if the details are important.

If 3 sine waves are turned on at once, there can be times when all three will add together for a maximum voltage.  The sum of the three voltage settings should therefore be kept below the overload point of about 0.6 volts.

\subsubsection{Noise}
Sig Gen \#4 produces Gaussian White noise.  There is an amplitude setting that corresponds to the 1-sigma voltage.  This is convenient since the noise power is this voltage squared and divided by 50.  What is not quite as convenient is to find the voltage setting that prevents overload, as Gaussian noise theoretically has no limit and the generator used here can achieve 12 times the 1-sigma voltage.  If you are looking for a high level of noise output, it is reasonable to set the 1-sigma point at 0.1 Volts overload (5-sigma = 0.50 Volts). For a 96 kHz sample rate, this will overload about every 20 seconds, on the average.  This will not be important for most experiments.

The noise power has a flat spectrum up to half the sample frequency.  The power density in Watts/Hz depends on the sample frequency as set by the Spectrum Analyzer.  For instance, if the 1-sigma noise level is 0.1 Volts, the power delivered to a 50-Ohm load is $0.1^2/50 = 0.0002$ Watts (0.2 milliWatts).   If the sample rate is 12 kHz the power is spread across 6 kHz so the power density is 0.2/6000 = 0.0000333 mW/Hz.  In dBm terms, this is -44.8 dBm/Hz.  As seen on the spectrum analyzer, with a noise bandwidth (for this sample rate) of 17.6 Hz, this will show about 17.6 * 0.0000333 = 0.000587 milliWatts per bin, or in dBm terms -32.3 dBm per bin.   This type of arithmetic allows setting the noise generator and the signal generators for any desired situation.

\subsubsection{Screen Control}
There is an overall signal generator LCD page, accessed by the "Signal Gens" menu button on the Home screen.  This gives a summary of the settings for all four generators and allows menu selection of an individual control screen for any of the four.  The first three screens, for Sig Gen \#1 through \#3, there is frequency selection to the nearest Hertz.  These use Up/Down touch buttons for control.  Similarly, all four have amplitude selection to the nearest millivolt.  The menu items at the bottom of the control screen allow the individual generator to be turned On or Off.  Two more  menu buttons allow selection of either a sine wave or a square wave for the 3 signal generators.

Note that all the control of these screens is also available from the Serial control using the USB plug.  Also note that over the Serial path the frequency can be controlled to the miliHertz, amplitude to the microVolt and the selections there also include triangle and two sawtooth waveforms.

\subsection{Discussion}
The settings of the four Audio Signal Generators (ASG) are independent, but they are all added together, appearing at the left hand port.  The frequencies are set in 1 Hz steps from 10 Hz to 40 kHz.  The amplitude has a maximum of about 0 dBm into a 50 Ohm load, which corresponds to 0.6324 Volts peak-to-peak. The screen setting is in peak-to-peak volts in order to not be awkward with non-sine waves. The waveform is selectable from the screen as either a sine or a square wave.
