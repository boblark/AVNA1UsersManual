\section{System Issues}
\label{sect:Sys}

\subsection{SIMULTANEITY}  We have limited input and output ports.  Combinations of Instruments and functions is limited by this.  Usually, this will be caught by the firmware if you try the impossible.  For instance, the four signal generators are turned off when the Vector Network Analyzer is running.  What we can do is to have multiple generators on at once.  For instance, the use of two (or three) generators for inter-modulation analysis is works well up to the overload point.  This, in part, is because the analog output amplifier uses much negative feedback and is quite linear.  Likewise combining signals and noise works well for setting specific S/N ratios, or other experiments.

\subsection{COHERENCY} If multiple waves are added together, and they are at non-harmonically related frequencies, the relative levels change at a fast rate.  In the other extreme, if the two waves are at the same frequency, they maintain a constant phase relationship. This can alter measurement results, depending on the phase. As implemented here, the phase relationship between two generators is a random  value.  For these reasons, you should not measure inter-modulation distortion with frequencies, such as 1000 and 2000 Hz.  Standard frequency pairs, such as 700 and 1900 Hz or 60 Hz and 7000 Hz support predictable results.

\subsection{EEPROM Update} We keep adding items to the permanent EEPROM memory in the Teensy (This is Flash memory set up to emulate EEPROM.).  At startup, the program checks the version that was used to write the EEPROM and adds nominal values for anything that was added after that version.  This prevents overwriting tune up values or the like.  If everything is up-to-date, the Serial monitor, at startup, should say, "\texttt{Loading EEPROM data; Turn-on EEPROM version was 83; EEPROM \newline Load of 536 bytes}."  The first time version 0.83 is run, there will be notes about the updating of the data.
